%!TEX TS-program = xelatex

% Шаблон документа LaTeX создан в 2018 году
% Алексеем Подчезерцевым
% В качестве исходных использованы шаблоны
% 	Данилом Фёдоровых (danil@fedorovykh.ru) 
%		https://www.writelatex.com/coursera/latex/5.2.2
%	LaTeX-шаблон для русской кандидатской диссертации и её автореферата.
%		https://github.com/AndreyAkinshin/Russian-Phd-LaTeX-Dissertation-Template

\documentclass[a4paper,14pt]{article}

\input{data/preambular.tex}
\begin{document} % конец преамбулы, начало документа
% \input{data/title.tex}
\tableofcontents
\pagebreak

\section{Введение}

Производственная практика пройдена в Институте проблем управления им. В. А. Трапезникова Российской академии наук (ИПУ РАН).

Целью прохождения производственной практики является закрепление и развитие профессиональных компетенций научно-исследовательской и проектной деятельности.

Для достижения поставленной цели потребовалось решить следующие задачи практики (в соответствии с программой практики):
\begin{itemize}
	\item Закрепление и расширение теоретических и практических знаний, полученных студентом в процессе обучения.

	\item Ознакомление со сферами деятельности организации.

	\item Получение навыков самостоятельной работы, а также работы в составе научно-исследовательских коллективов.

	\item Работа над проектом по созданию детектора наличия медицинской маски на человеке.

	\item Обработка полученных материалов и оформление отчета о прохождении практики.
\end{itemize}

В ходе прохождения производственной практики для выполнения задания языком программирования был выбран python3, для обработки изображений применялась библиотека fastai, графический интерфейс был создан при помощи фреймворка tkinter.

\pagebreak
\section{Содержательная часть}

\subsection{Описание профессиональных задач студента}

Содержание практики (вопросы, подлежащие изучению):

\begin{enumerate}
	
	\item Прохождение инструктажа по технике безопасности на предприятии.
	
	\item Исследование текущего состояния систем проверки наличия медицинской маски на человеке.
	
	\item Подготовка дата-сета для обучения нейронной сети для проверки наличия медицинской маски на человеке.
	
	\item Разработка архитектуры нейронной сети для проверки наличия медицинской маски на человеке.
	
	\item Разработка детектора наличия медицинской маски на человеке.
	
	\item Обучение нейронной сети для проверки наличия медицинской маски на человеке.
	
	\item Сбор, обобщение и анализ полученных в ходе производственной практики материалов и подготовка отчета по практике.
	
\end{enumerate}

В последнее вемя проблема определения наличия медицинской маски на лице человека стала особо актуальной.
Особенно, после введения всевозможных мер, требующих от предприятий и организаций повышенной внимательности и контроля за соблюдением санэпидемиологических требований.

Так как тема новая, аналогов данного детектора не много.
Удалось найти лишь одно готовое решение данной проблемы, это \href{https://fisher.cvizi.com/#solutions}{CVizi Fisher: Masks}.
Компания занимается видеоаналитикой для наблюдения и контроля появления людей в определенных зонах с определенными условиями. 
Несомненным достоинством данного решения является простота использования, но есть и недостатки.
Первый недостаток в том, что эта услуга платная и, помимо единоразовой платы за оборудование, придется платить и за абонентскую плату, 15 рублей в день за каждую камеру.
Второй недостаток в том, что все это решение в "коробке", и у пользователя нет возможности как-то добавить или изменить функционал,
%Третий недостаток в том, что все сырые данные с камер поступают на специальный микрокомпьютер, оттуда в обработанном виде на сервера э компании.
Кроме этого, нигде не упоминается о качестве модели, используемой в приложении.




Для обработки видео и изображений лучше всего подходят нейронные сети.

\subsection{Описание выполнения пунктов}

выполнение

\pagebreak
\section{Заключение}

Здесь будет заключение

\pagebreak
\section{Приложения	}

Здесь будут приложения

\newpage 
\renewcommand{\refname}{{\normalsize Список использованных источников}} 
\centering 
\begin{thebibliography}{9} 
	\addcontentsline{toc}{section}{\refname} 
%	\bibitem{Verilog} Thomas D., Moorby P. The Verilog Hardware Description Language. – Springer Science \& Business Media, 2008.
%	\bibitem{Quartus} Антонов А., Филиппов А., Золотухо Р. Средства системной отладки САПР Quartus II //Компоненты и технологии. – 2008. – №. 89.
\end{thebibliography}

\end{document} % конец документа