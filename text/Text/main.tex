%!TEX TS-program = xelatex

% Шаблон документа LaTeX создан в 2018 году
% Алексеем Подчезерцевым
% В качестве исходных использованы шаблоны
% 	Данилом Фёдоровых (danil@fedorovykh.ru) 
%		https://www.writelatex.com/coursera/latex/5.2.2
%	LaTeX-шаблон для русской кандидатской диссертации и её автореферата.
%		https://github.com/AndreyAkinshin/Russian-Phd-LaTeX-Dissertation-Template

\documentclass[a4paper,14pt]{article}

\input{data/preambular.tex}
\begin{document} % конец преамбулы, начало документа
% \input{data/title.tex}
\tableofcontents
\pagebreak

\section{Введение}

Производственная практика пройдена в Институте проблем управления им. В. А. Трапезникова Российской академии наук (ИПУ РАН).

Целью прохождения производственной практики является закрепление и развитие профессиональных компетенций научно-исследовательской и проектной деятельности.

Для достижения поставленной цели потребовалось решить следующие задачи практики (в соответствии с программой практики):
\begin{itemize}
	\item Закрепление и расширение теоретических и практических знаний, полученных студентом в процессе обучения.

	\item Ознакомление со сферами деятельности организации.

	\item Получение навыков самостоятельной работы, а также работы в составе научно-исследовательских коллективов.

	\item Работа над проектом по созданию детектора наличия медицинской маски на человеке.

	\item Обработка полученных материалов и оформление отчета о прохождении практики.
\end{itemize}

В ходе прохождения производственной практики языком программирования был python3, для обработки изображений применялась библиотека fastai, графический интерфейс был создан при помощи фреймворка tkinter.

\pagebreak
\section{Содержательная часть}

\subsection{Описание профессиональных задач студента}

описание

\subsection{Описание выполнения пунктов}

выполнение

\pagebreak
\section{Заключение}

Здесь будет заключение

\pagebreak
\section{Приложения	}

Здесь будут приложения

\newpage 
\renewcommand{\refname}{{\normalsize Список использованных источников}} 
\centering 
\begin{thebibliography}{9} 
	\addcontentsline{toc}{section}{\refname} 
%	\bibitem{Verilog} Thomas D., Moorby P. The Verilog Hardware Description Language. – Springer Science \& Business Media, 2008.
%	\bibitem{Quartus} Антонов А., Филиппов А., Золотухо Р. Средства системной отладки САПР Quartus II //Компоненты и технологии. – 2008. – №. 89.
\end{thebibliography}

\end{document} % конец документа