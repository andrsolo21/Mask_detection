%!TEX TS-program = xelatex

% Шаблон документа LaTeX создан в 2018 году
% Алексеем Подчезерцевым
% В качестве исходных использованы шаблоны
% 	Данилом Фёдоровых (danil@fedorovykh.ru) 
%		https://www.writelatex.com/coursera/latex/5.2.2
%	LaTeX-шаблон для русской кандидатской диссертации и её автореферата.
%		https://github.com/AndreyAkinshin/Russian-Phd-LaTeX-Dissertation-Template

\documentclass[a4paper,14pt]{article}


%%% Работа с русским языком
\usepackage[english,russian]{babel}   %% загружает пакет многоязыковой вёрстки
\usepackage{fontspec}      %% подготавливает загрузку шрифтов Open Type, True Type и др.
\defaultfontfeatures{Ligatures={TeX},Renderer=Basic}  %% свойства шрифтов по умолчанию
\setmainfont[Ligatures={TeX,Historic}]{Times New Roman} %% задаёт основной шрифт документа
\setsansfont{Comic Sans MS}                    %% задаёт шрифт без засечек
\setmonofont{Courier New}
\usepackage{indentfirst}
\frenchspacing

\renewcommand{\epsilon}{\ensuremath{\varepsilon}}
\renewcommand{\phi}{\ensuremath{\varphi}}
\renewcommand{\kappa}{\ensuremath{\varkappa}}
\renewcommand{\le}{\ensuremath{\leqslant}}
\renewcommand{\leq}{\ensuremath{\leqslant}}
\renewcommand{\ge}{\ensuremath{\geqslant}}
\renewcommand{\geq}{\ensuremath{\geqslant}}
\renewcommand{\emptyset}{\varnothing}

%%% Дополнительная работа с математикой
\usepackage{amsmath,amsfonts,amssymb,amsthm,mathtools} % AMS
\usepackage{icomma} % "Умная" запятая: $0,2$ --- число, $0, 2$ --- перечисление

%% Номера формул
%\mathtoolsset{showonlyrefs=true} % Показывать номера только у тех формул, на которые есть \eqref{} в тексте.
%\usepackage{leqno} % Нумерация формул слева	

%% Перенос знаков в формулах (по Львовскому)
\newcommand*{\hm}[1]{#1\nobreak\discretionary{}
	{\hbox{$\mathsurround=0pt #1$}}{}}

%%% Работа с картинками
\usepackage{graphicx}  % Для вставки рисунков
\graphicspath{{images/}}  % папки с картинками
\setlength\fboxsep{3pt} % Отступ рамки \fbox{} от рисунка
\setlength\fboxrule{1pt} % Толщина линий рамки \fbox{}
\usepackage{wrapfig} % Обтекание рисунков текстом

%%% Работа с таблицами
\usepackage{array,tabularx,tabulary,booktabs} % Дополнительная работа с таблицами
\usepackage{longtable}  % Длинные таблицы
\usepackage{multirow} % Слияние строк в таблице
\usepackage{float}% http://ctan.org/pkg/float

%%% Программирование
\usepackage{etoolbox} % логические операторы


%%% Страница
\usepackage{extsizes} % Возможность сделать 14-й шрифт
\usepackage{geometry} % Простой способ задавать поля
\geometry{top=20mm}
\geometry{bottom=20mm}
\geometry{left=20mm}
\geometry{right=10mm}
%
%\usepackage{fancyhdr} % Колонтитулы
% 	\pagestyle{fancy}
%\renewcommand{\headrulewidth}{0pt}  % Толщина линейки, отчеркивающей верхний колонтитул
% 	\lfoot{Нижний левый}
% 	\rfoot{Нижний правый}
% 	\rhead{Верхний правый}
% 	\chead{Верхний в центре}
% 	\lhead{Верхний левый}
%	\cfoot{Нижний в центре} % По умолчанию здесь номер страницы

\usepackage{setspace} % Интерлиньяж
\onehalfspacing % Интерлиньяж 1.5
%\doublespacing % Интерлиньяж 2
%\singlespacing % Интерлиньяж 1

\usepackage{lastpage} % Узнать, сколько всего страниц в документе.

\usepackage{soul} % Модификаторы начертания

\usepackage{hyperref}
\usepackage[usenames,dvipsnames,svgnames,table,rgb]{xcolor}
\hypersetup{				% Гиперссылки
	unicode=true,           % русские буквы в раздела PDF
	pdftitle={Отчет по практике},   % Заголовок
	pdfauthor={Солодянкин А.А.},      % Автор
	pdfsubject={Распознавание масок},      % Тема
	pdfcreator={Солодянкин А.А.}, % Создатель
	pdfproducer={Солодянкин А.А.}, % Производитель
	pdfkeywords={keyword1} {key2} {key3}, % Ключевые слова
	colorlinks=true,       	% false: ссылки в рамках; true: цветные ссылки
	linkcolor=black,          % внутренние ссылки
	citecolor=black,        % на библиографию
	filecolor=magenta,      % на файлы
	urlcolor=black           % на URL
}
\makeatletter 
\def\@biblabel#1{#1. } 
\makeatother
\usepackage{cite} % Работа с библиографией
%\usepackage[superscript]{cite} % Ссылки в верхних индексах
%\usepackage[nocompress]{cite} % 
\usepackage{csquotes} % Еще инструменты для ссылок

\usepackage{multicol} % Несколько колонок

\usepackage{tikz} % Работа с графикой
\usepackage{pgfplots}
\usepackage{pgfplotstable}

% ГОСТ заголовки
\usepackage[font=small]{caption}
%\captionsetup[table]{justification=centering, labelsep = newline} % Таблицы по правобу краю
%\captionsetup[figure]{justification=centering} % Картинки по центру


\newcommand{\tablecaption}[1]{\addtocounter{table}{1}\small \begin{flushright}\tablename \ \thetable\end{flushright}%	
\begin{center}#1\end{center}}

\newcommand{\imref}[1]{рис.~\ref{#1}}

\usepackage{multirow}
\usepackage{spreadtab}
\newcolumntype{K}[1]{@{}>{\centering\arraybackslash}p{#1cm}@{}}


\usepackage{xparse}
\usepackage{fancyvrb}

\RecustomVerbatimCommand{\VerbatimInput}{VerbatimInput}
{
	fontsize=\footnotesize    
}

\newcolumntype{?}[1]{!{\vrule width #1}}

\usepackage{tocloft}
\renewcommand{\cftsecleader}{\cftdotfill{\cftdotsep}}
\begin{document} % конец преамбулы, начало документа
	 \begin{titlepage}
	\begin{center}
		ПРАВИТЕЛЬСТВО РОССИЙСКОЙ ФЕДЕРАЦИИ \\
 		ФЕДЕРАЛЬНОЕ  ГОСУДАРСТВЕННОЕ АВТОНОМНОЕ \\
		ОБРАЗОВАТЕЛЬНОЕ УЧРЕЖДЕНИЕ ВЫСШЕГО ОБРАЗОВАНИЯ\\
		«НАЦИОНАЛЬНЫЙ ИССЛЕДОВАТЕЛЬСКИЙ УНИВЕРСИТЕТ\\
		«ВЫСШАЯ ШКОЛА ЭКОНОМИКИ»
	\end{center}
	
	\begin{center}
		\textbf{Московский институт электроники и математики}
		
		\textbf{Им. А.Н.Тихонова НИУ ВШЭ}
		
		\vspace{2ex}
		
		\textbf{Департамент компьютерной инженерии}
	\end{center}
	\vspace{1ex}	
	
	\vspace{1ex}
	\begin{center}

		
	\end{center}	

	\vspace{2ex}
	\vfill
	
	\vspace{2ex}
	
	\begin{flushright}
	
	\end{flushright}

	\vspace{5ex}
	\begin{center}
		Москва \the\year \, г.
	\end{center}
	
\end{titlepage}
\addtocounter{page}{1}
	
	\tableofcontents
	
	\pagebreak
	\section{Индивидуальное задание}
	
	\pagebreak
	
	-
	
	\pagebreak
	
	\section{Дневник практики}
	
	\pagebreak
	
	-
	
	\pagebreak
	
	
	\section{Введение}
	
	Производственная практика пройдена в Институте проблем управления им. В. А. Трапезникова Российской академии наук (ИПУ РАН).
	
	Целью прохождения производственной практики является закрепление и развитие профессиональных компетенций научно-исследовательской и проектной деятельности.
	
	Для достижения поставленной цели потребовалось решить следующие задачи практики (в соответствии с программой практики):
	\begin{itemize}
		\item Закрепление и расширение теоретических и практических знаний, полученных студентом в процессе обучения.
		
		%\item Ознакомление со сферами деятельности организации.
		
		\item Получение навыков самостоятельной работы, а также работы в составе научно-исследовательских коллективов.
		
		\item Работа над проектом по созданию детектора наличия медицинской маски на человеке.
		
		\item Обработка полученных материалов и оформление отчета о прохождении практики.
	\end{itemize}
	
	В ходе прохождения производственной практики для выполнения задания языком программирования был выбран python3, для обработки изображений применялась библиотека fastai, графический интерфейс был создан при помощи фреймворка tkinter.
	
	\pagebreak
	\section{Содержательная часть}
	
	\subsection{Описание профессиональных задач студента}
	
	Содержание практики (вопросы, подлежащие изучению):
	
	\begin{enumerate}
		
		\item Прохождение инструктажа по технике безопасности на предприятии.
		
		\item Исследование текущего состояния систем проверки наличия медицинской маски на человеке.
		
		\item Подготовка дата-сета для обучения нейронной сети для проверки наличия медицинской маски на человеке.
		
		\item Разработка архитектуры нейронной сети для проверки наличия медицинской маски на человеке.
		
		\item Разработка детектора наличия медицинской маски на человеке.
		
		\item Обучение нейронной сети для проверки наличия медицинской маски на человеке.
		
		\item Сбор, обобщение и анализ полученных в ходе производственной практики материалов и подготовка отчета по практике.
		
	\end{enumerate}
	
	В последнее время проблема определения наличия медицинской маски на лице человека стала особо актуальной.
	Особенно, после введения всевозможных мер, требующих от предприятий и организаций повышенной внимательности и контроля за соблюдением санэпидемиологических требований.
	
	Так как тема новая, аналогов данного детектора не много.
	Удалось найти лишь одно готовое решение данной проблемы, это \href{https://fisher.cvizi.com/#solutions}{CVizi Fisher: Masks}.
	Компания занимается видеоаналитикой для наблюдения и контроля появления людей в определенных зонах с определенными условиями. 
	Несомненным достоинством данного решения является простота использования, но есть и недостатки.
	Первый недостаток в том, что эта услуга платная и, помимо единоразовой платы за оборудование, придется платить и за абонентскую плату, 15 рублей в день за каждую камеру.
	Второй недостаток в том, что все это решение в <<коробке>>, и у пользователя нет возможности как-то добавить или изменить функционал,
	%Третий недостаток в том, что все сырые данные с камер поступают на специальный микрокомпьютер, оттуда в обработанном виде на сервера э компании.
	Кроме этого, нигде не упоминается о качестве модели, используемой в приложении.
	
	Для обучения модели необходим размеченный дата-сет с изображениями.
	
	\subsubsection{Принцип работы сверточных нейронных сетей}
	
	Для обработки видео и изображений лучше всего подходят сверточные нейронные сети, так как они лучше справляются с однородными данными изображения.
	
	Основой сверточных нейронных сетей являются сверточные слои.
	Им на вход подается массив матриц с данными.
	В случае цветного изображения на вход первого сверточного слоя попадают 3 матрицы, соответствующие 3 цветовым каналам RGB, размеры этих матриц соответствуют размерам исходного изображения.
	Слой свертки включает для каждого канала свой фильтр, ядро свертки которого обрабатывает предыдущий слой по фрагментам, суммируя результаты поэлементного произведения каждого фрагмента.
	Ядро свертки представляет из себя небольшую матрицу обычно размером 3×3, числовые коэффициенты внутри матрицы подбираются в процессе обучения.
	Пример работы продемонстрирован на рис. \ref{fig:0-144563-628298}.
	
	Далее, скалярный результат каждой свертки попадает в функцию активации.
	Функция активации может быть разной, главное, чтобы она обладала свойством нелинейности.
	Пример такой функции -- ReLU, вычисляется на как $f(x) = max(0,x)$.
	
	Слой пулинга представляет собой нелинейное уплотнение карты признаков, при этом группа пикселей (обычно размера 2×2) уплотняется до одного пикселя, проходя нелинейное преобразование.
	Наиболее употребительна при этом функция максимума.
	Преобразования затрагивают непересекающиеся прямоугольники или квадраты, каждый из которых ужимается в один пиксель, при этом выбирается пиксель.
	Операция пулинга позволяет существенно уменьшить пространственный объём изображения. 
	Пример работы продемонстрирован на рис. \ref{fig:maxpooling}.
	%Самая популярная нелинейная операция пулинга -- пулинг максимума.
	
	После нескольких прохождений свёртки изображения и уплотнения с помощью пулинга система перестраивается от конкретной сетки пикселей с высоким разрешением к более абстрактным картам признаков.
	После сверточных слоев остается большое число каналов, хранящих небольшое число данных, которые воспринимаются как различные очертания, выявленные из исходного изображения.
	Эти данные объединяются и передаются в обычную полносвязную сеть.
	
	Наиболее популярны способом обучения является метод обучения с учителем -- метод обратного распространения ошибки.
	
	\subsubsection{Архитектуры сверточных нейронных сетей}
	
	Наиболее популярным архитектурами нейронных сетей для обработки изображений на данный момент являются:
	%https://www.reg.ru/blog/stehnfordskij-kurs-lekciya-9-arhitektury-cnn/
	\begin{itemize}
		\item AlexNet
		\item VGGNet
		\item GoogleNet
		\item ResNet
	\end{itemize}
	
	AlexNet -- прорывная для своего времени архитектура сверточной нейронной сети.
	AlexNet была первой свёрточной нейросетью, выигравшей соревнование по классификации ImageNet в 2012 году с результатом 16.4\%.
	Её архитектура состоит из пяти сверточных слоёв, между которыми располагаются pooling-слои и слои нормализации, а завершают нейросеть три полносвязных слоя.
	
	В 2014 году VGGNet достигла более чем в два раза лучшего результата по сравнению с AlexNet.
	Основная идея VGG-архитектур — использование большего числа слоёв с фильтрами меньшего размера.
	
	GoogleNet — ещё более глубокая архитектура с 22 слоями.
	Целью Google было разработать нейросеть с наибольшей вычислительной эффективностью.
	Для этого они придумали так называемый модуль Inception — вся архитектура состоит из множества таких модулей, следующих друг за другом.
	Идея основного модуля Inception заключается в том, что он сам по себе является небольшой локальной сетью.
	Вся его работа состоит в параллельном применении нескольких фильтров на исходное изображение.
	Данные фильтров объединяются, и создаётся выходной сигнал, который переходит на следующий слой.
	Пример архитектуры GoogleNet рис. \ref{fig:googlenet}.
	
	В 2015 году ResNet произвела настоящую революцию глубины нейросетей.
	Она состояла из 152 слоёв и снизила процент ошибок до 3,57\% в соревновании классификации ImageNet.
	Это сделало её почти в два раза эффективнее GoogleNet.
	
	Обычная нейростеть при сильном увеличении количества слоев показывает результат хуже как при обучении, так и при тестировании.
	Создатели ResNet предположили, что загвоздка кроется в оптимизации — более глубокие модели гораздо хуже поддаются настройке.
	Тогда они решили не складывать слои друг на друга для изучения отображения нужной функции напрямую, а использовать остаточные блоки, которые пытаются «подогнать» это отображение.
	Нейросеть <<перепрыгивает>> через некоторые слои.
	Они больше не содержат признаков и используются для нахождения остаточной функции $H(x) = F(x) + x$ вместо того, чтобы искать $H(x)$ напрямую.
	Нейросеть состоит из большого стека одинаковых остаточных блоков, каждый из которых имеет два свёрточных слоя 3×3.
	Периодически число фильтров удваивается, а их размерность уменьшается с шагом 2 (/ 2 в каждом измерении).
	В самом начале архитектуры присутствует дополнительный свёрточный слой.
	Также у ResNet нет полносвязных слоёв в конце — используется только один слой с выходными классами.
	С увеличением числа слоёв для уменьшения размерности изображения применяются точно такие же дополнительные слои, как и в GoogleNet. 
	В результате экспериментов с ResNet выяснилось, что очень глубокие сети действительно можно обучить без ухудшения точности.
	Нейросеть достигла наименьшей ошибки в задачах классификации, которая превзошла даже человеческий результат.
	Пример архитектуры ResNet рис. \ref{fig:resnetarc}.
	
	Результаты популярных архитектур на соревновании ImageNet представлены на рис. \ref{fig:graphlayers}.
	
	%Лучшее качество среди приведенных нейросетей показала архитектура ResNet.
	
	\subsubsection{Обучение нейросети}
	
	Обучить нейросеть можно двумя способами.
	
	Для первого способа обучения нейронной сети необходимо задать все коэффициенты случайными значениями, а потом подбирать при помощи обратного распространения ошибки.
	Данный способ требует много времени и вычислительных мощностей.
	Кроме этого, необходим очень большой размеченный датасет для обучения.
	
	Второй способ называется fine-tuning (тонкая настройка).
	Для него нужны куда меньшие вычислительные мощности и меньший размер датасета, при этом можно достичь лучшего качества.
	Суть способа заключается в том, чтобы взять уже готовую, предобученную модель и дообучить ее на наших данных.
	У модели будут уже подобранные коэффициенты для извлечения нужных очертаний объектов.
	Остается лишь дообучить модель на новом промаркированном датасете.
	%Для лучшего качества и меньших затрат по времени на обучение лучше взять предобученную модель и дообучить её на нашем дата-сете.
	
	%\subsubsection{Метрики качества}
	%https://habr.com/ru/company/ods/blog/328372/
	%При определении наличия маски решается задача классификации.
	%Для задач классификации есть несколько метрик качества:
	
	
	
	\subsection{Описание выполнения пунктов}
	
	\subsubsection{Обучение нейросети}
	
	Для того, чтобы приступить к обучению нейросети, необходим датасет с размеченными изображениями.
	
	В интернете был найден \href{https://github.com/UniversalDataTool/coronavirus-mask-image-dataset}{репозиторий}, в котором решается похожая задача по распознаванию маски на лице человека.
	В репозитории есть csv файл с промаркированными ссылками на изображения.
	
	Для обучения модели была использована библиотека \href{https://docs.fast.ai/}{fast.ai}.
	Библиотека fast.ai упрощает обучение качественных нейронных сетей с использованием современных передовых практик.
	Она основана на исследованиях в области передового опыта глубинного обучения, включая поддержку <<из коробки>> для моделей обработки изображений, текста и коллаборативной фильтрации.
	
	В качестве архитектуры нейронной сети выбрана архитектура ResNet, так как она показала лучшие результаты в соревновании классификации ImageNet.
	В библиотеке fast.ai для архитектуры ResNet есть обученные модели с разным числом слоев: 18, 34, 50, 101 и 152.
	Детальная архитектура представлена на рис. \ref{fig:resnet-architectures-34-101}.
	
	Были протестированы все варианты нейронной сети.
	Нейросеть из 18 слоев занимает меньше всего места и работает быстрее остальных, но качество уступает моделям с большим числом слоев.
	Качество остальных 4 моделей примерно одинаковое, при этом скорость обработки изображения падает с увеличением числа слоев.
	Была использована модель с 34 слоями, так как она оптимальна с точки зрения качества и времени обработки изображения.
	Матрицы ошибок для протестированных моделей приведены на рис. \ref{fig:resnet18} -- \ref{fig:resnet152}.
	
	\subsubsection{Создание демонстрационного варианта}
	
	Для демонстрации работы модели было создана программа на Python, которая обрабатывает видео с веб камеры и выводит его и вероятность нахождения маски на изображении на экран.
	Демонстрационный вариант работает следующим образом:
	Кадр с веб камеры берется при помощи библиотеки openCV, и хранится в памяти в качестве трехмерного тензора.
	Далее этот тензор оправляется в обученную модель, которая на выход дает вероятность нахождения маски в кадре.
	Для исключения выбросов итоговый результат берется как среднее последних 3 измерений.
	В завершении итоговый результат и кадр обновляются в графическом окне.
	
	Графическая часть реализована при помощи библиотеки tkinter.
	Итоговый вид окна представлен на рис. \ref{fig:test0}.
	
	
	\subsubsection{Тестирование}
	
	На рис. \ref{fig:test0} -- \ref{fig:test70} представлены скриншоты демонстрационного окна.
	На рис. \ref{fig:test0} и рис. \ref{fig:test100} все просто, где нет маски вероятность близка к нулю, где маска хорошо надета вероятность 100\%.
	На рис. \ref{fig:test70} маска надета не полностью и нейросеть выдает уже меньший результат. 
	
	\subsubsection{Дальнейшее развитие}
	
	Данная тема имеет много возможностей для развития.
	
	Первое, что необходимо сделать, это добавить модуль для решения задач нахождения объектов (object detection).
	Лицо человека в маске или без необходимо сначала выделить и после этого определять наличие маски на выделенном участке изображения.
	Данное дополнение может сильно улучшить качество работы детектора масок.
	
	Как дополнение можно сделать детектор и других средств индивидуальной защиты, например, перчаток.
	Также можно определять качество защитной маски.
	
	\pagebreak
	\section{Заключение}
	
	В ходе практики, был изучен принцип работы сверточной нейронной сети,  рассмотрены самые популярные архитектуры сверточных нейронных сетей и подходы к их обучению.
	
	Были приобретены навыки по поиску необходимого датасета, по работе со сверточными нейронными сетями, их дообучению и применению в реальном приложении.
	Помимо этого, были получены навыки по работе с графическим интерфейсом в python.
	
	По окончанию практики была достигнута главная цель - применение теоретических знаний, полученных в процессе обучения, при решении реальных задач.
	
	А также приобретены навыки и опыт практической работы.
	Данная практика является хорошим практическим опытом для дальнейшей самостоятельной деятельности.
	
	\pagebreak
	\section{Приложения}
	
	С кодом можно ознакомиться по ссылке: \href{https://github.com/andrsolo21/Mask_detection}{https://github.com/andrsolo21/Mask\_detection}.
	
	\begin{figure}[H]
		\centering
		\includegraphics[width=0.7\linewidth]{image/0-144563-628298}
		\caption{Пример работы свертки}
		\label{fig:0-144563-628298}
	\end{figure}
	
	\begin{figure}[H]
		\centering
		\includegraphics[width=0.7\linewidth]{image/Max_pooling}
		\caption{Пример работы пулинг слоя}
		\label{fig:maxpooling}
	\end{figure}
	
	\begin{figure}[H]
		\centering
		\includegraphics[width=0.7\linewidth]{image/google_net}
		\caption{Архитектура GoogleNet}
		\label{fig:googlenet}
	\end{figure}
	
	
	\begin{figure}[H]
		\centering
		\includegraphics[width=0.7\linewidth]{image/resnet_arc}
		\caption{Пример архитектуры ResNet}
		\label{fig:resnetarc}
	\end{figure}
	
	
	\begin{figure}[H]
		\centering
		\includegraphics[width=0.7\linewidth]{image/graph_layers}
		\caption{Результаты популярных архитектур на соревновании ImageNet}
		\label{fig:graphlayers}
	\end{figure}
	
	\begin{figure}[H]
		\centering
		\includegraphics[width=0.7\linewidth]{image/resnet-architectures-34-101}
		\caption{Архитектуры сетей ResNet}
		\label{fig:resnet-architectures-34-101}
	\end{figure}
	
	
	\begin{figure}[H]
		\centering
		\includegraphics[width=0.4\linewidth]{image/resnet18}
		\caption{Матрица ошибок для resnet18}
		\label{fig:resnet18}
	\end{figure}
	
	\begin{figure}[H]
		\centering
		\includegraphics[width=0.4\linewidth]{image/resnet34}
		\caption{Матрица ошибок для resnet34}
		\label{fig:resnet34}
	\end{figure}
	
	\begin{figure}[H]
		\centering
		\includegraphics[width=0.4\linewidth]{image/resnet50}
		\caption{Матрица ошибок для resnet50}
		\label{fig:resnet50}
	\end{figure}
	
	\begin{figure}[H]
		\centering
		\includegraphics[width=0.4\linewidth]{image/resnet101}
		\caption{Матрица ошибок для resnet101}
		\label{fig:resnet101}
	\end{figure}
	
	\begin{figure}[H]
		\centering
		\includegraphics[width=0.4\linewidth]{image/resnet152}
		\caption{Матрица ошибок для resnet152}
		\label{fig:resnet152}
	\end{figure}
	
	\begin{figure}[H]
		\centering
		\includegraphics[width=0.7\linewidth]{image/test_0}
		\caption{Тестирование без маски}
		\label{fig:test0}
	\end{figure}
	
	\begin{figure}[H]
		\centering
		\includegraphics[width=0.7\linewidth]{image/test_100}
		\caption{Тестирование в маске}
		\label{fig:test100}
	\end{figure}
	
	\begin{figure}[H]
		\centering
		\includegraphics[width=0.7\linewidth]{image/test_70}
		\caption{Тестирование с недоконца надетой маской}
		\label{fig:test70}
	\end{figure}
	
	
	%\newpage 
	%\renewcommand{\refname}{{\normalsize Список использованных источников}} 
	%\centering 
	%\begin{thebibliography}{9} 
	%	\addcontentsline{toc}{section}{\refname} 
	%	\bibitem{Verilog} Thomas D., Moorby P. The Verilog Hardware Description Language. – Springer Science \& Business Media, 2008.
	%	\bibitem{Quartus} Антонов А., Филиппов А., Золотухо Р. Средства системной отладки САПР Quartus II //Компоненты и технологии. – 2008. – №. 89.
	%\end{thebibliography}
	
\end{document} % конец документа